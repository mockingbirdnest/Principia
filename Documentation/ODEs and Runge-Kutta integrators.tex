\documentclass[10pt, a4paper, twoside]{basestyle}
\usepackage[Mathematics]{semtex}

%%%% Shorthands.

%%%% Title and authors.

\title{%
\textdisplay{%
An Introduction to Runge-Kutta Integrators}%
}
\author{Robin~Leroy (eggrobin)}
\begin{document}
\maketitle
In this post I shall assume understanding of the concepts described in chapter~8 (Motion) as well as  chapter~11 (Vectors) of Richard Feynmann's \emph{Lectures on Physics}.

\section{Motivation}
We want to be able to predict the position $\vs\of t$ as a function of time of a spacecraft (without engines) around a fixed planet of mass $M$. In order to do this, we recall that the velocity is given by
\[\vv = \deriv t \vs\]
and the acceleration by
\[\va = \deriv t \vv = \deriv[2] t \vs\text.\]
We assume the mass of the spacecraft is constant and that the planet sits at the origin of our reference frame. Newton's law of universal gravitation tells us that the magnitude (the length) of the acceleration vector will be \[
a=\frac{G M}{s^2}\text,
\]
where $s$ is the length of $\vs$.
and that the acceleration will be directed towards the planet, so that\[
\va=-\frac{G M}{s^2} \frac{\vs}{s}\text.
\]
We don't really care about the specifics, but we see that this is a function of $\vs$. We'll write it $\va\of\vs$.
Putting it all together we could rewrite this as
\[\deriv[2] t \vs = \va\of\vs\]
and go ahead and solve this kind of problem, but we don't like having a second derivative. Instead we just write down both equations,
\[
\begin{cases}
\deriv t \vs = \vv \\
\deriv t \vv = \va\of\vs
\end{cases}\text.
\]
Let us define a vector $\vy$ with 6 entries instead of 3,
\[\vy = \tuple{\vs, \vv} = \tuple{s_x, s_y, s_z, v_x, v_y, v_z}\text.\]
Similarly, define a function $\vf$ as follows:
\[\vf\tuple{\vy} = \tuple{\vv, \va\of\vs}\text.\]
Our problem becomes
\[\deriv t \vy = \tuple{\deriv t \vs, \deriv t \vv} = \tuple{\vv, \va\of\vs} = \vf\of\vy\text.\]
So we have gotten rid of that second derivative.

\section{Ordinary differential equations}
We are interested computing solutions to equations of the form
\[\deriv t \vy = \vf\of{\vy,t}\text.\]
Such an equation is called an \emph{ordinary differential equation} (\textsc{ode}). The function $\vf$ is called the \emph{right-hand side} (\textsc{rhs}).

Recall that if the right-hand side didn't depend on $\vy$, the answer would be the integral,
\[\deriv t \vy = \vf\of t \Implies \vy = \int{} \vf\of t \diffd t\text.\]
In the case where the right-hand side doesn't depend on $t$ (but depends on $\vy$), as was the case in the previous section, the equation becomes
\[\deriv t \vy = \vf\of{\vy}\text.\]
We call such a right-hand side \emph{autonomous}.

In order to compute a particular solution (a particular trajectory of our spacecraft), we need to define some initial conditions (the initial position and velocity of our spacecraft) at $t=t_0$. We write them as\[
y\of{t_0} = y_0\text.
\]
The \textsc{ode} together with the initial conditions form the \emph{initial value problem} (\textsc{ivp})
\[
\begin{cases}
\deriv t \vy = \vf\of{\vy,t} \\
y\of{t_0} = y_0[
\end{cases}\text.
\]
\section{Euler's method}
As we want to actually store the solution in a computer, we can't compute $\vy\of t$ for all values of $t$. Instead we approximate $\vy\of{t}$ at discrete time steps.

How do we compute the first point $\vy_1$, the approximation for $\vy\of{t_1}$? By definition of the derivative, we get \[
\lim_{\conv{\increment t}{0}} \frac{\vy\of{t_0+\increment t}}{\increment t} =  \vf\of{\vy_0,t_0}\text.
\]
This means that if we take a sufficiently small $\increment t$, set $t_1 = t_0 + \increment t$, we have
\[
\frac{\vy\of{t_1}}{\increment t} \approx \vf\of{\vy_0,t_0}\text,
\]
where the approximation gets better as $\increment t$ gets smaller.
Our first method for approxmimating the solution is therefore to compute\[
y_1 = \vf\of{\vy_0,t_0} \increment t\text.\]
For the rest of the solution, we just repeat the same method, yielding \[
y_{n+1} = \vf\of{\vy_n,t_n} \increment t\text.\]
This is called \emph{Euler's method}, after Swiss mathematician Leonhard Euler (1707--1783).

We are want to know two things: how good our approximation is, and how much we need to reduce $\increment t$ in order to make it better.
In order to do that, we use Taylor's theorem.

\paragraph*{Taylor's theorem}
Recall that if $\deriv t \vy=\TimeDerivative\vy\of t$ is constant, $\vy\of{t+\increment t} = \vy\of{t_0} + \SecondTimeDerivative\vy\of{t_0}$. If $\deriv[2] t \vy=\TimeDerivative\vy\of t$

\end{document}

