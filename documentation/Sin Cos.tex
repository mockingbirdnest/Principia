\documentclass[10pt, a4paper, twoside]{basestyle}

\usepackage{tikz}
\usetikzlibrary{cd}

\usepackage[Mathematics]{semtex}
\usepackage{chngcntr}
\counterwithout{equation}{section}

%%%% Shorthands.
\DeclareMathOperator{\bias}{\mathit{bias}}
\DeclareMathOperator{\ULP}{\mathfrak u}
\DeclareMathOperator{\mant}{\mathfrak m}
\DeclareMathOperator{\expn}{\mathfrak e}
\DeclareMathOperator{\truncate}{\StandardSymbol{Tr}}

% Rounding brackets will be heavily nested, and reading the nesting depth is critically important,
% so we make them grow for readability.
\newcommand{\round}[1]{\doubleSquareBrackets*{#1}}
\newcommand{\roundTowardZero}[1]{\doubleSquareBrackets{#1}_0}
\newcommand{\roundTowardPositive}[1]{\doubleSquareBrackets{#1}_+}
\newcommand{\roundTowardNegative}[1]{\doubleSquareBrackets{#1}_-}
\newcommand{\hex}[1]{{_{16}}\mathrm{#1}}
\newcommand{\bin}[1]{{_{2}}\mathrm{#1}}

%%%% Title and authors.

\title{An Implementation of Sin and Cos Using Gal's Accurate Tables}
\date{\printdate{2025-02-02}}
\author{Pascal~Leroy (phl)}
\begin{document}
\maketitle
\begin{sloppypar}
\noindent
This document describes the implementation of functions \texttt{Sin} and \texttt{Cos} in Principia.  The goals of that implementation are to be portable (including to machines that do not have a fused multiply-add instruction), achieve good performance, and ensure correct rounding.
\end{sloppypar}

\section*{Overview}
The implementation follows the ideas described by \cite{GalBachelis1991} and uses accurate tables produced by the method presented in \cite{StehléZimmermann2005}.  It guarantees correct rounding with a high probability.  In circumstances where it cannot guarantee correct rounding, it falls back to the (slower but correct) implementation provided by the CORE-MATH project \cite{SibidanovZimmermannGlondu2022} \cite{ZimmermannSibidanovGlondu2024}.  More precisely, the algorithm proceeds through the following steps:
\begin{itemize}[nosep]
\item perform argument reduction using Cody and Waite's algorithm in double precision (see \cite[379]{MullerBrisebarreDeDinechinJeannerodLefevreMelquiondRevolStehleTorres2010});
\item if argument reduction loses too many bits (i.e., the argument is close to a multiple of $\frac{\Pi}{2}$), fall back to \texttt{cr\_sin} or \texttt{cr\_cos};
\item otherwise, uses accurate tables and a polynomial approximation to compute \texttt{Sin} or \texttt{Cos} with extra accuracy;
\item if the result has a ``dangerous rounding configuration'' (as defined by \cite{GalBachelis1991}), fall back to \texttt{cr\_sin} or \texttt{cr\_cos};
\item otherwise return the rounded result of the preceding computation.
\end{itemize}
In this document we assume a base-2 floating-point number system with $M$ significand bits\footnote{In \texttt{binary64}, $M = 53$.} similar to the IEEE formats.  We define a real  function $\mant$ and an integer function $\expn$ denoting the \emph{significand} and \emph{exponent} of a real number, respectively:
\[
x = ±\mant\of{x} \times 2^{\expn\of{x}} \qquad\text{with}\qquad 2^{M-1} \leq \mant\of{x} \leq 2^M - 1
\]
Note that this representation is unique.  Furthermore, if $x$ is a floating-point number, $\mant\of{x}$ is an integer.

The distance between $1$ and the next larger floating-point number is:
\[
\ge_M \DefineAs 2^{1-M}
\]
and the distance between $1$ and the next smaller floating-point number is $\frac{\ge_M}{2}$.
The \emph{unit of the last place} of $x$ is defined as:
\[
\ULP\of{x} \DefineAs 2^{\expn\of{x}}
\]
In particular, $\ULP\of{1} = \ge_M$ and:
\[
\frac{2 x}{\ge_M} < \frac{x}{2^M - 1} \leq \ULP\of{x} \leq \frac{x}{2^{M - 1}} = \frac{x}{\ge_M}
\]

We ignore the exponent bias, overflow and underflow as they play no role in this discussion.

Finally, for error analysis we use the accuracy model of \cite{Higham2002}, equation (2.4): unless otherwise indicated, $\gd_i$ is a roundoff factor such that $\gd_i < u = \ge_M / 2 = 2^{-M}$ (see pages 37 and 38).  We also use $\gq_n$ and $\gg_n$ with the same meaning as in \cite{Higham2002}, lemma 3.1.

\section*{Approximation of $\frac{\Pi}{2}$}
To perform argument reduction, we need to build approximations of $\frac{\Pi}{2}$ with extra accuracy and analyse the circumstances under which they may be used and the errors that they entail on the reduced argument.

We start by defining the truncation function $\truncate\of{\gk, z}$ which clears the last $\gk$ bits of the significand of $z$:
\[
\truncate\of{\gk, z} \DefineAs \floor{2^{-\gk} \mant \of{z}} 2^{\gk} \ULP\of{z}
\]
The definition of the floor function implies:
\[
0 \leq z - \truncate\of{\gk, z} < 2^{\gk} \ULP\of{z}
\]
Furthermore if the bits that are being truncated start with $k$ zeros we have the stricter inequality:
\begin{equation}
0 \leq z - \truncate\of{\gk, z} < 2^{\gk'} \ULP\of{z} \quad \text{with} \quad \gk' = \gk - k
\label{eqntruncerror}
\end{equation}
This leads to the following upper bound for the unit of the last place of the truncation error:
\[
\ULP\of{z - \truncate\of{\gk, z}} < \frac{2^{\gk'} \ULP\of{z}}{\mant\of{z - \truncate\of{\gk, z}}} \leq 2^{\gk' - M + 1} \ULP\of{z}
\]
which can be made more precise by noting that the function $\ULP$ is always a power of $2$:
\begin{equation}
\ULP\of{z - \truncate\of{\gk, z}} = 2^{\gk' - M} \ULP\of{z}
\label{eqnulptr}
\end{equation}

\subsubsection*{Two-Term Approximation}
In this scheme we approximate $\frac{\Pi}{2}$ as the sum of two floating-point numbers:
\[
\frac{\Pi}{2} ≃ C_1 + \gd C_1
\]
which are defined as:
\begin{equation*}
\begin{dcases}
C_1 &\DefineAs \truncate\of{\gk_1, \frac{\Pi}{2}} \\
\gd C_1 &\DefineAs \round{\frac{\Pi}{2} - C_1}
\end{dcases}
\end{equation*}
Equation (\ref{eqnulptr}) becomes:
\[
\ULP\of{\frac{\Pi}{2} - C_1} = 2^{\gk'_1 - M} \ULP\of{\frac{\Pi}{2}}
\]
where $\gk'_1 \leq \gk_1$ accounts for any leading zeroes in the bits of $\frac{\Pi}{2}$ that are being truncated.
The absolute error on the two-term approximation is therefore:
\begin{align}
\abs{\frac{\Pi}{2} - C_1 - \gd C_1} \leq \frac{1}{2} \ULP\of{\frac{\Pi}{2} - C} = 2^{\gk'_1 - M - 1} \ULP\of{\frac{\Pi}{2}} = 2^{\gk'_1 - 2 M}
\label{eqnpitwoterms}
\end{align}
In other words, we have a representation with a significand that has effectively $2 M - \gk'_1$ bits and is such that multiplying $C_1$ by an integer less than or equal to $2^{\gk_1}$ is exact.

\subsubsection*{Three-Term Approximation}
In this scheme we approximate $\frac{\Pi}{2}$ as the sum of three floating-point numbers:
\[
\frac{\Pi}{2} ≃ C_2 + C'_2 + \gd C_2
\]
which are defined as:
\begin{equation*}
\begin{dcases}
C_2 &\DefineAs \truncate\of{\gk_2, \frac{\Pi}{2}} \\
C'_2 &\DefineAs \truncate\of{\gk_2, \frac{\Pi}{2} - C_2} \\
\gd C_2 &\DefineAs \round{\frac{\Pi}{2} - C_2 - C'_2}
\end{dcases}
\end{equation*}
Applying equation (\ref{eqnulptr}) to the definition of $C_2$ yields:
\[
\ULP\of{\frac{\Pi}{2} - C_2} = 2^{\gk'_2 - M} \ULP\of{\frac{\Pi}{2}}
\]
where $\gk'_2 \leq \gk_2$ accounts for any leading zeroes in the bits of $\frac{\Pi}{2}$ that are being truncated.  Similarly, applying equation (\ref{eqnulptr}) to the definition of $C'_2$ yields:
\begin{align*}
\ULP\of{\frac{\Pi}{2} - C_2 - C'_2} &= 2^{\gk''_2 - M} \ULP\of{\frac{\Pi}{2} - C_2} \\
&= 2^{\gk'_2 + \gk''_2 - 2 M} \ULP\of{\frac{\Pi}{2}}
\end{align*}
where $\gk''_2 \leq \gk_2$ accounts for any leading zeroes in the bits of $\frac{\Pi}{2} - C_2$ that are being truncated.  Note that normalization of the significand of $\frac{\Pi}{2} - C_2$ effectively drops the zeroes at positions $\gk_2$ to $\gk'_2$ and therefore the computation of $C'_2$ applies to a significand aligned on position $\gk'_2$.

In other words, we have a representation with a significand that has effectively $3 M - \gk'_2 - \gk''_2$ bits and is such that multiplying $C_2$ and $C'_2$ by an integer less than or equal to $2^{\gk_2}$ is exact.

\section*{Argument Reduction}
Given an argument $x$, the purpose of argument reduction is to compute a pair of floating-point numbers $\pa{\hat x, \gd \hat x}$ such that:
\[
\begin{dcases}
\hat x + \gd \hat x ≅ x \pmod{\frac{\Pi}{2}} \\
\hat x \;\text{is approximately in}\; \intclos{-\frac{\Pi}{4}}{\frac{\Pi}{4}} \\
\abs{\gd \hat x} < \ULP\of{\hat x} 
\end{dcases}
\]

\subsection*{Argument Reduction for Small Angles}
If $\abs x < \round{\frac{\Pi}{4}}$ then $\hat x = x$ and $\gd \hat x = 0$.
\subsection*{Argument Reduction Using the Two-Term Approximation}
If $\abs x \leq 2^{\gk_1} \round{\frac{\Pi}{2}}$ we compute:
\[
\begin{dcases}
n &= \iround{\round{x \round{\frac{2}{\Pi}}}} \\
y &= x - n \; C_1 \\
\gd y &= \round{n \; \gd C_1} \\
\hat x &= \round{y - \gd y} \\
\gd \hat x &= \pa{y - \hat x} - \gd y
\end{dcases}
\]
Let's first show that $\abs n \leq 2^{\gk_1}$. :
\begin{align*}
\abs x &\leq 2^{\gk_1} \frac{\Pi}{2} \pa{1 + \gd_1} \\
\abs n &\leq \iround{2^{\gk_1} \frac{\Pi}{2} \pa{1 + \gd_1} \frac{2}{\Pi} \pa{1 + \gd_2} \pa{1 + \gd_3}} \\
&\leq \iround{2^{\gk_1} \pa{1 + \gg_3}}
\end{align*}
As long as $2^{\gk_1} \gg_3$ is small enough (less that $1/2$), the rounding cannot cause $n$ to exceed $2^{\gk_1}$.

The product $n \; C_1$ is exact thanks to the $\gk_1$ trailing zeroes of $C_1$.  The subtraction $x - n \; C_1$ is exact by Sterbenz's Lemma.  Finally, the last two steps form a compensated summation so that $\hat x + \gd \hat x = y + \gd y$.

To compute the overall error on argument reduction, first remember that, from equation (\ref{eqnpi}) we have:
\[
C_1 + \gd C_1 = \frac{\Pi}{2} + \gd_5 \quad \text{with} \quad \abs{\gd_5} \leq 2^{\gk'_1 - M - 1} \ULP\of{\frac{\Pi}{2}}
\]
The error computation proceeds as follows:
\begin{align*}
y + \gd y &= x - n \; C_1 - n \; \gd C_1 \pa{1 + \gd_4} \\
&= x - n \pa{C_1 + \gd C_1} - n \; \gd C_1 \; \gd_4 \\
&= x - n \frac{\Pi}{2} -n \pa{\gd_5 + \gd C_1 \; \gd_4}
\end{align*}
from which we can deduce an upper bound on the absolute error:
\[
\abs{y + \gd y - \pa{x - n \frac{\Pi}{2}}} < 2^{\gk_1} 2^{\gk_2} \ULP\of{\frac{\Pi}{2}} \pa{2^{- M - 1} + 2^{-M}} = \frac{3}{2} 2^{\gk_1 + \gk'_1 - M} \ULP\of{\frac{\Pi}{2}}
\]
where we have used the upper bound for $\gd C_1$ given by equation (\ref{eqntruncerror}).

If we want $\hat x + \gd \hat x$ to have $\gk_3$ extra bits of accuracy, we must have:
\[
\frac{3}{2} 2^{\gk_1 + \gk'_1 - M} \ULP\of{\frac{\Pi}{2}} \leq 2^{-\gk_3} \abs{\ULP\of{\hat x}} \leq 2^{-\gk_3 - M + 1} \abs{\hat x}
\]
which leads to the following condition on the reduced angle:
\[
\abs{\hat x} \geq \frac{3}{2} 2^{\gk_1 + \gk'_1 + \gk_3} \ULP\of{\frac{\Pi}{2}} = \frac{3}{2} 2^{1 + \gk_1 + \gk'_1 + \gk_3 - M}
\]

If we choose $\gk_1 = 8$ we find that $\gk'_1 = 5$ (because there are three consecutive zeroes at this location in the significand of $\frac{\Pi}{2}$) and the desired accuracy is obtained as long as $\abs{\hat x} \geq 3 \times 2^{-21} ≃ 7.2 \times 10^{-7}$.

\subsection*{Argument Reduction Using the Three-Term Approximation}
\section*{Accurate Tables and Their Generation}
\section*{Computation of the Functions}
\subsection*{Sin}
\subsubsection*{Near Zero}
For $\hat x$ near zero we evaluate:
\begin{align*}
\widehat{x^2} &= \round{{\hat x}^2} = {\hat x}^2 \pa{1 + \gd_1}\\
\widehat{x^3} &= \round{\hat x \; \widehat{x^2}} = {\hat x}^3 \pa{1 + \gd_1} \pa{1 + \gd_2}\\
\hat p &= \round{a \widehat{x^2} + b} = \pa{a {\hat x}^2 \pa{1 + \gd_1} + b} \pa{1 + \gd_3}\\
s\of{x} &\DefineAs \hat x + \round{\round{\widehat{x^3} \hat p} + \gd \hat x} \\
&= \hat x + \pa{{\hat x}^3 \pa{1 + \gd_1} \pa{1 + \gd_2} \pa{a {\hat x}^2 \pa{1 + \gd_1} + b} \pa{1 + \gd_3} \pa{1 + \gd_4} + \gd \hat x}\pa{1 + \gd_5} \\
&= \hat x + a {\hat x}^3 \pa{1 + \gq_5} + b {\hat x}^5 \pa{1 + \gq_4} + \gd \hat x \pa{1 + \gd_5}
\end{align*}
\end{document}
