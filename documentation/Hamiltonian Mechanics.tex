\documentclass[10pt, a4paper, twoside]{basestyle}
\usepackage[Mathematics]{semtex}

%%%% Shorthands.

%%%% Title and authors.

\title{%
\textdisplay{%
An Introduction Hamiltonian Mechanics}%
}
\author{Robin~Leroy (eggrobin)}
\begin{document}
\maketitle
In this post I shall assume understanding of the concepts described in
chapter~4 (Conservation of Energy), chapter~8 (Motion) as well as sections
11--4 and 11--5 (Vectors and Vector algebra) of chapter~11 of Richard
Feynmann's \emph{Lectures on Physics}.

It is the continuation of my \emph{Introduction to Runge-Kutta Integrators},
but it does not rely on the concepts described in that post.

\section{Motivation}
We have previously seen how to compute the evolution of physical systems while
keeping the buildup of error in check. However, the error will still build up
over time. We would like to ensure that fundamental properties of the physical
system are preserved. For instance, we'd like a low strongly-bound orbit not to
turn into an escape trajectory (or a reentry) over time: we need conservation
of energy.

In order to make an integrator that conserves energy, it is helpful to look at
physics from a viewpoint where the conservation of energy is the fundamental
hypothesis, rather than a consequence of the application of some forces.

\section{Gravitation from a Newtonian viewpoint}
Recall the formulation by forces of the gravitational $N$-body problem: each
body $i$, located at position $\vq_i$ exerts a force $\vF_{ij}$ on every other body $j$, located at $\vq_j$ whose magnitude is
\[
F_{ij} = \frac{G m_i m_j}{\norm{\vq_i - \vq_j}^2}.
\]
The force on $j$ is directed toward $i$, so
\[
\vF_{ij} = F_{ij} \frac{\vq_i - \vq_j}{\norm{\vq_i - \vq_j}} =
\frac{G m_i m_j}{\norm{\vq_i - \vq_j}^2}
    \frac{\vq_i - \vq_j}{\norm{\vq_i - \vq_j}} =
\frac{G m_i m_j}{\norm{\vq_i - \vq_j}^3}\pa{\vq_i - \vq_j}.
\]
Adding up all the forces exerted on $j$ by the bodies $1, \dotsc, j-1, j+1,
\dotsc, N$, we get the total force $\vF_j$ on $j$,
\begin{align*}
\vF_j &=
\vF_{1 j} + \vF_{2 j} + \dotsb + \vF_{j-1\, j} + \vF_{j+1\, j} + \vF_{N\, j} \\
&= \sum{i\in\intclos 1 N \setminus \set{j}} \vF_{i j}
\end{align*}
This force then changes the velocity of $j$ according to Newton's second law,
\[
\deriv t {\vv_j} = \frac{\vF_j}{m_j}.
\]
What is the energy of that $N$-body system? Part of it is the energy due to the motion of the bodies, the so-called \emph{kinetic}\footnote{From ancient Greek \emph{κινεῖν}, to move.} energy $T$, \[
T = \sum{j=1}[N] \frac12 m_j v_j^2.
\]
\end{document}