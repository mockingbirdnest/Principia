\documentclass[10pt, a4paper, oneside]{basestyle}
\usepackage[Mathematics]{semtex}

%%%% Shorthands.

%%%% Title and authors.

\title{%
\textdisplay{%
Explicit representation of Legendre polynomials%
}%
}
\author{Pascal~Leroy (phl)}
\begin{document}
\maketitle
\noindent
According to Wikipedia (\url{https://en.wikipedia.org/wiki/Legendre_polynomials}), the Legendre polynomials have the following representation in terms of monomials, which is immediate from the recursion formula:
\begin{equation*}
P_n(x) = 2^n \sum{k = 0}[n] x^k \binom{n}{k} \binom{\frac{n + k - 1}{2}}{n}
\end{equation*}
In this formula, the second binomial coefficient is actually nontrivial to evaluate because its upper index is not necessarily integral, and its lower index is greater than its upper index.
First note that, because of the parity of the Legendre polynomials, the terms where $k$ and $n$ do not have the same value modulo 2 are 0.  Therefore the above can be rewritten as:
\begin{equation*}
P_n(x) = 2^n \sum{\substack{k = 0\\ k = n mod 2}}[n] x^k \binom{n}{k} \binom{\frac{n + k - 1}{2}}{n}
\end{equation*}

\end{document}