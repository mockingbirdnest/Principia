\documentclass[10pt, a4paper, oneside]{basestyle}
\usepackage[Mathematics]{semtex}

%%%% Shorthands.

%%%% Title and authors.

\title{%
\textdisplay{%
Documentation for the symplectic methods%
}%
}
\author{Robin~Leroy (eggrobin)}
\begin{document}
\maketitle
This document expands on the comments at the beginning of\\
\texttt{\small integrators/symplectic\_runge\_kutta\_nyström\_integrator.hpp}.

\section{Differential equations.}
Recall that the equations solved by this class are
\begin{align}
\tuple{\vq,\vp}\der &=
\vX\of{\vq, \vp, t} = \vA\of{\vq, \vp} + \vB\of{\vq, \vp, t}
\quad\parbox{.4\linewidth}{with $\exp h\vA$ and $\exp h\vB$ known and
$\commutator{\vB}{\commutator{\vB}{\commutator{\vB}{\vA}}}=\nullvec$;} \\
&\text{the above equation, with $\exp h\vA = h\vA$, $\exp h\vB = h\vB$,
and $\vA$ and $\vB$ known;} \\
\vq\dder &= -\matM^{-1} \grad_\vq V\of{\vq, t}\text. \label{rkn}
\end{align}

\section{Relation to Hamiltonian mechanics.}
The third equation above is a reformulation of Hamilton's
equations with a Hamiltonian of the form
\begin{equation}
H\of{\vq,\vp,t} = \frac{1}{2}\Transpose{\vp}\matM^{-1}\vp + V\of{\vq, t}\text,
\end{equation}
where $\vp = \matM\vq\der$.

\section{A remark on non-autonomy.}
Most treatments of these integrators write these differential equations as well
as the corresponding Hamiltonian in an autonomous version, thus
$\vX = \vA(\vq, \vp) + \vB(\vq, \vp)$ and
$H\of{\vq,\vp,t} = \frac{1}{2}\Transpose{\vp}\matM^{-1}\vp + V\of{\vq}$.
It is however possible to incorporate time, by considering it as an
additional variable:\[
\tuple{\vq,\vp,t}\der =
\vX\of{\vq, \vp, t} =
\tuple{\vA\of{\vq, \vp}, 1} +
\tuple{\vB\of{\vq, \vp, t}, 0}\text.\]
		
For equations of the form (\ref{rkn}) it remains to be shown that Hamilton's
equations with quadratic kinetic energy and a time-dependent potential satisfy
$\commutator{\vB}{\commutator{\vB}{\commutator{\vB}{\vA}}}=\nullvec$.
We introduce $t$ and its conjugate momentum $\gcp$ to the phase space,
and write
\[
\tilde\vq=\tuple{\vq, t}\text,\quad
\tilde\vp=\tuple{\vp, \gcp}\text,\quad
L\of{\tilde\vp} = \frac{1}{2}\Transpose{\vp}\matM^{-1}\vp + \gcp\text.
\]
(\ref{rkn}) follows from Hamilton's equations with\[
H\of{\tilde\vq,\tilde\vp} =
L\of{\tilde\vp} + V\of{\tilde\vq} =
\frac{1}{2}\Transpose{\vp}\matM^{-1}\vp + \gcp + V\of{\vq, t}
\]		
since we then get $t\der = 1$.		
The desired property follows from the following lemma:
\begin{lemma}
Let $L\of{\tilde\vq,\tilde\vp}$ be a quadratic polynomial in $\tilde\vp$,
$V\of{\tilde\vq}$ a smooth function, $\vA=\Poisson\placeholder L$, and
$\vB=\Poisson\placeholder V$.
Then\[
\commutator{\vB}{\commutator{\vB}{\commutator{\vB}{\vA}}}=\nullvec\text.\]
\end{lemma}
\begin{proof}
It suffices to show that $\Poisson V{\Poisson V{\Poisson L V}} = 0$. It is
immediate that every term in that expression will contain a third order
partial derivative in the $\tilde p_i$ of $L$, and since $L$ is quadratic
all such derivatives vanish.
\end{proof}

\end{document}